\documentclass{article}
\usepackage{graphicx}
\usepackage{lipsum} % for dummy text, you can remove this package
\usepackage{hyperref}
\usepackage{placeins} % for \FloatBarrier
\title{Data Analysis Application Report}
\author{Maria Balasopoulou AM inf2021150, Theodora Matsaridou AM inf2021136}
\date{\today}

\begin{document}

\maketitle

\section{Link of the application}
http://192.168.1.15:3838 

\section{Introduction}
The Data Analysis Application presented in this report serves as a tool for users to analyze data from CSV or Excel files. Its main functionalities include data loading, variable selection, data visualization, classification using Decision Tree or Logistic Regression, and clustering using K-Means or Hierarchical Clustering.

\section{Design Overview}
The application is built using RShiny, with a user interface (UI) and server components. The UI consists of multiple tabs, each dedicated to a specific functionality. Users can load their data file, select variables, visualize data, perform classification, and conduct clustering analysis. The server components handle the logic behind each functionality and interact with the UI elements.

\section{Implementation Details}
The implementation of the application utilizes various R packages, including \texttt{shiny}, \texttt{rpart}, \texttt{ggplot2}, \texttt{dplyr}, and \texttt{cluster}. Key implementation details for each module are as follows:

\subsection{Data Loading}
The data loading module allows users to upload CSV or Excel files. Upon file selection, the application reads the data using \texttt{read.csv} or \texttt{read\_excel} functions from the \texttt{readr} package.

\begin{verbatim}
inFile <- input$file
if (is.null(inFile))
  return(NULL)
df <- read.csv(inFile$datapath)
\end{verbatim}

\begin{figure}
    \centering
    \includegraphics[width=1.3\textwidth]{loadfile.png}
    \caption{load file example}
    \label{fig:Data Loading}
\end{figure}

\FloatBarrier % Ensures all figures before this point are placed

\subsection{Variable Selection}
Users can select the X and Y variables for data visualization, classification, and clustering. The application dynamically generates dropdown menus based on the columns of the loaded dataset.

\begin{verbatim}
# X variable selection
  output$x_var_selector <- renderUI({
    req(data())
    selectInput("x_var", "Choose X variable:", choices = colnames(data()))
  })  
# Y variable selection
  output$y_var_selector <- renderUI({
    req(data())
    selectInput("y_var", "Choose Y variable:", choices = colnames(data()))
  })
\end{verbatim}

\begin{figure}
    \centering
    \includegraphics[width=1.0\textwidth]{variables_selection.png}
    \caption{variables selection example}
    \label{fig:Variable Selection}
\end{figure}

\FloatBarrier % Ensures all figures before this point are placed

\subsection{Data Visualization}
The data visualization module generates a scatter plot using \texttt{ggplot2} to visualize the relationship between the selected variables.

\begin{verbatim}
ggplot(data(), aes(x = !!sym(x_var), y = !!sym(y_var))) +
  geom_point() +
  labs(x = x_var, y = y_var) +
  theme_minimal()
\end{verbatim}

\begin{figure}
    \centering
    \includegraphics[width=1.3\textwidth]{2D Visualization.png}
    \caption{2D Visualization example}
    \label{fig:Data Visualization}
\end{figure}
\FloatBarrier
\subsection{Classification}
Users can choose between Decision Tree and Logistic Regression algorithms for classification. The models are trained using \texttt{rpart} and \texttt{glm} functions, respectively.

\begin{verbatim}
# Classification with Decision Tree or Logistic Regression
  accuracy <- reactiveValues(dt = NULL, lr = NULL)
  
  observeEvent(input$classify_button, {
    req(input$file)
    req(input$x_var)
    req(input$y_var)
    req(input$classification_algorithm)
    
    if (input$classification_algorithm == "Decision Tree") {
      data_processed <- data() %>%
        mutate(y_category = cut(.data[[input$y_var]], breaks = 3, labels = c("Low", "Medium", "High")))
      dt_model <- rpart(y_category ~ ., data = data_processed)
      prediction <- predict(dt_model, newdata = data_processed, type = "class")
      accuracy$dt <- sum(prediction == data_processed$y_category) / nrow(data_processed)
      output$classification_result <- renderPrint({
        cat("Decision Tree Model Summary:\n")
        print(summary(dt_model))
        cat("\nAccuracy:", accuracy$dt, "\n")
      })
    } else if (input$classification_algorithm == "Logistic Regression") {
      threshold <- 0.5
      df <- data() %>%
        mutate(y_category = cut(.data[[input$y_var]], breaks = 3, labels = c("Low", "Medium", "High")))
      lr_model <- glm(formula = as.formula(paste("y_category ~ .")), data = df, family = binomial, maxit = 1000)
      prediction <- predict(lr_model, newdata = df, type = "response")
      prediction <- ifelse(prediction > threshold, "High", ifelse(prediction < threshold, "Low", "Medium"))
      accuracy$lr <- sum(prediction == df$y_category) / nrow(df)
      output$classification_result <- renderPrint({
        cat("Logistic Regression Model Summary:\n")
        print(summary(lr_model))
        cat("\nAccuracy:", accuracy$lr, "\n")
      })
    }
  })
  
  # Comparison of algorithm accuracies
  output$accuracy_comparison <- renderPrint({
    req(input$classify_button)
    
    if (!is.null(accuracy$dt) && !is.null(accuracy$lr)) {
      if (accuracy$dt > accuracy$lr) {
        cat("Decision Tree has higher accuracy.\n")
      } else if (accuracy$dt < accuracy$lr) {
        cat("Logistic Regression has higher accuracy.\n")
      } else {
        cat("Both algorithms have the same accuracy.\n")
      }
    }
  })
\end{verbatim}

\begin{figure}
    \centering
    \includegraphics[width=1.3\textwidth]{decision_tree.png}
    \caption{decision tree example 1}
    \label{fig:Classification}
\end{figure}

\begin{figure}
    \centering
    \includegraphics[width=1.3\textwidth]{decision_tree2.png}
    \caption{decision tree example 2}
    \label{fig:Classification}
\end{figure}

\begin{figure}
    \centering
    \includegraphics[width=1.3\textwidth]{decision_tree3.png}
    \caption{decision tree example 3}
    \label{fig:Classification}
\end{figure}

\begin{figure}
    \centering
    \includegraphics[width=1.3\textwidth]{logistic_regression1.png}
    \caption{logistic regression example 1}
    \label{fig:Classification}
\end{figure}

\begin{figure}
    \centering
    \includegraphics[width=1.3\textwidth]{logistic_regression2.png}
    \caption{logistic regression example 2}
    \label{fig:Classification}
\end{figure}

\FloatBarrier

\subsection{Clustering}
K-Means and Hierarchical Clustering algorithms are available for clustering analysis. They are implemented using \texttt{kmeans} and \texttt{hclust} functions, respectively.

\begin{verbatim}
# Clustering with K-Means or Hierarchical Clustering
  silhouette_scores <- reactiveValues(kmeans = NULL, hclust = NULL)
  
  observeEvent(input$run_clustering, {
    req(input$file)
    req(input$x_var)
    req(input$y_var)
    req(input$clustering_algorithm)
    
    if (input$clustering_algorithm == "K-Means") {
      kmeans_result <- kmeans(data()[, c(input$x_var, input$y_var)], centers = 3)
      cluster_labels <- kmeans_result$cluster
      silhouette_score <- silhouette(cluster_labels, dist(data()[, c(input$x_var, input$y_var)]))
      silhouette_scores$kmeans <- mean(silhouette_score[, "sil_width"])
      output$clustering_result <- renderPrint({
        cat("K-Means Clustering Summary:\n")
        print(kmeans_result)
        cat("\nSilhouette Score:", silhouette_scores$kmeans, "\n")
      })
    } else if (input$clustering_algorithm == "Hierarchical Clustering") {
      hclust_result <- hclust(dist(data()[, c(input$x_var, input$y_var)]))
      cluster_labels <- cutree(hclust_result, k = 3)
      silhouette_score <- silhouette(cluster_labels, dist(data()[, c(input$x_var, input$y_var)]))
      silhouette_scores$hclust <- mean(silhouette_score[, "sil_width"])
      output$clustering_result <- renderPrint({
        cat("Hierarchical Clustering Summary:\n")
        print(hclust_result)
        cat("\nSilhouette Score:", silhouette_scores$hclust, "\n")
      })
    }
    
    # Comparison of algorithm silhouette scores
    if (!is.null(silhouette_scores$kmeans) && !is.null(silhouette_scores$hclust)) {
      output$silhouette_comparison <- renderPrint({
        if (silhouette_scores$kmeans > silhouette_scores$hclust) {
          cat("K-Means has higher Silhouette Score.\n")
        } else if (silhouette_scores$kmeans < silhouette_scores$hclust) {
          cat("Hierarchical Clustering has higher Silhouette Score.\n")
        } else {
          cat("Both algorithms have the same Silhouette Score.\n")
        }
      })
    }
  })
\end{verbatim}

\FloatBarrier

\begin{figure}
    \centering
    \includegraphics[width=1.3\textwidth]{k-means.png}
    \caption{K means example }
    \label{fig:Clustering}
\end{figure}

\begin{figure}
    \centering
    \includegraphics[width=1.3\textwidth]{hierarchical_clustering.png}
    \caption{hierarchical clustering example }
    \label{fig:Clustering}
\end{figure}
\FloatBarrier
\subsection{Info}
the info tab is dedicated to the presentation 
information about the application, how it works and the development team.
\begin{figure}
    \centering
    \includegraphics[width=1.3\textwidth]{info.png}
    \caption{info example }
    \label{fig:Info}
\end{figure}

\FloatBarrier

\textbf{}\section{Results of Analyses}
The application provides results for classification accuracies and silhouette scores for clustering. These results are displayed to the user after performing the respective analyses.

\section{Conclusions}
The Data Analysis Application offers users a versatile platform for exploring and analyzing their data. Through its intuitive interface and powerful analytical capabilities, users can gain valuable insights and make data-driven decisions. Future enhancements could include additional algorithms for classification and clustering, as well as improved visualization options.

\section{Contributions of Team Members}
 Maria focused on the implementation of data loading, variable selection, and clustering functionalities, while Theodora primarily worked on data visualization, classification, and result presentation. Both team members contributed to the design, testing, and documentation of the application.

\section{UML Diagram}
 the UML diagram depicting the application and interface architecture user interface
\FloatBarrier

\begin{figure}[htbp]
    \centering
    \includegraphics[width=1.3\textwidth]{UML_Diagram.png}
    \caption{UML diagram }
    \label{fig:UML Diagram}
\end{figure}
\FloatBarrier


\section{ Software Release Lifecycle Waterfall}
Software Release Lifecycle with the Waterfall Model for Data Analysis Application

1. Requirements Analysis:
Using CSV and Excel files as a data source.Selecting variables for
analysis.Creating scatter plots.Performing classification using Decision Tree and Logistic Regression.Performing clustering using K-Means and Hierarchical Clustering.Comparative analysis of results and algorithms. Requirements Documentation:We document all requirements and functions in a
detailed requirements document, which includes:Data loading and preview
functions.Variable selection capabilities for analysis.Categorization and clustering functions with corresponding results and comparisons.

2 System Design:
We analyze and design the details for the individual system components, such as:UI components for file selection, variable selection and buttons for running algorithms.Functions for reading CSV/Excel files and data processing.Integration of categorization and clustering algorithms.

3. ImplementationCoding:
Develop the application based on the design. This
includes:Developing the UI using the Shiny framework.Implementing the functions for reading and processing data from CSV/Excel files.Integrating the categorization (Decision Tree, Logistic Regression) and clustering (K-Means, Hierarchical Clustering) algorithms.Creating graphs with ggplot2.Internal Testing:Testing the code to ensure that all functions are implemented correctly.

4. Testing (Testing)System Testing:
Comprehensive testing of the application to ensure that
all functions work correctly and work together smoothly.Testing of file loading and data preview.Testing of variable selection and graph generation.Testing of the accuracy and performance of the categorization and clustering algorithms.Confirming that results and comparisons are displayed correctly.Bug Fixes: Correcting any bugs found during testing.

5. Integration and Maintenance System Integration:
We integrate the application into a production environment, ensuring that it is accessible to end users.Monitoring and Support. We provide support to users and collect feedback for improvements.Maintenance:
Troubleshooting and improving the application based on user feedback and new
requirements as they arise.

Conclusions: 
Using the Waterfall model to develop the data analytics application allows for a structured and linear approach, which is appropriate when requirements are well defined from the start. This process ensures that each phase is fully completed before moving on to the next, thus ensuring the quality and stability of the application before it is made available to the general public.

\section{Github}
link : https://github.com/mariampal11/TL-ergasia.git
In this link is the github repository. it contains the application code , the LaTeX code , the dockerfile , the docker-compose.yml and a docker-steps file which explains in detail what steps we followed and at the end it has the link of our application.
\end{document}
